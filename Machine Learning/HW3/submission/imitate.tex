\documentclass[twoside,11pt]{homework}

\coursename{COMS 4771 Machine Learning (Spring 2015)} % DON'T CHANGE THIS

\studname{Name Surname}    % YOUR NAME GOES HERE
\studmail{uni@columbia.edu}% YOUR UNI GOES HERE
\hwNo{1}                   % THE HOMEWORK NUMBER GOES HERE
\collab{djh2164,jbh2019}   % THE UNI'S OF STUDENTS YOU DISCUSSED WITH

\begin{document}
\maketitle

\section*{Problem 1}

% YOUR SOLUTION GOES HERE

% SOME EXAMPLE LATEX CODE BELOW (DON'T INCLUDE IN YOUR ACTUAL SUBMISSION!)
Examples of blackboard and calligraphic letters: $\bbR^d \supset
\bbS^{d-1}$, $\cC \subset \cB$.
Examples of bold-faced letters (perhaps suitable for matrix and
vectors):
\begin{equation}
  L(\vx,\vlambda) = f(\vx) - \innerprod{\vlambda,\vA\vx-\vb} .
  \label{eq:lagrangian}
\end{equation}
\newcommand\var{\ensuremath{\operatorname{var}}}%
Example of a custom-defined math operator:
\[
  \var(X) = \bbE X^2 - (\bbE X)^2 .
\]
Example of references: the Lagrangian is given in
Eq.~\eqref{eq:lagrangian}, and Theorem~\ref{thm:euclid} is
interesting.
Example of adaptively-sized parentheses:
\[
  \left(\prod_{i=1}^n x_i\right)^{1/n}
  + \left(\prod_{i=1}^n y_i\right)^{1/n}
  \leq
  \left(\prod_{i=1}^n (x_i + y_i)\right)^{1/n}
  .
\]
Example of aligned equations:
\begin{align}
  \Pr(X = 1 \,|\, Y = 1)
  & = \frac{\Pr(X = 1 \,\wedge\, Y = 1)}{\Pr(Y = 1)}
  \notag \\
  & = \frac{\Pr(Y = 1 \,|\, X = 1) \cdot \Pr(X = 1)}{\Pr(Y = 1)}
  .
  \label{eq:bayes-rule}
\end{align}
Example of a theorem:
\begin{theorem}[Euclid]
  \label{thm:euclid}
  There are infinitely many primes.
\end{theorem}
\begin{proof}[Euclid's proof]
  There is at least one prime, namely $2$.
  Now pick any finite list of primes $p_1, p_2, \dotsc, p_n$.
  It suffices to show that there is another prime not on the list.
  Let $p := \prod_{i=1}^n p_i + 1$, which is not any of the primes on
  the list.
  If $p$ is prime, then we're done.
  So suppose instead that $p$ is not prime.
  Then there is prime $q$ which divides $p$.
  If $q$ is one of the primes on the list, then it would divide $p -
  \prod_{i=1}^n p_i = 1$, which is impossible.
  Therefore $q$ is not one of the $n$ primes in the list, so we're
  done.
\end{proof}

\section*{Problem 2}

% YOUR SOLUTION GOES HERE

\section*{Problem 3}

% YOUR SOLUTION GOES HERE

\section*{Problem 4}

% YOUR SOLUTION GOES HERE

\section*{Problem 5}

% YOUR SOLUTION GOES HERE

\end{document} 